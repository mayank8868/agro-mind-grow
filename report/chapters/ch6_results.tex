\chapter{Results and Discussion}

\section{User Interface Results}
The user interface was designed with a focus on simplicity and accessibility for farmers.

\subsection{Home Page / Dashboard}
The dashboard serves as the central hub for the application. It provides quick access to all modules and displays a summary of the current weather.

\begin{figure}[H]
    \centering
    \includegraphics[width=1.0\textwidth]{homepage}
    \caption{AgroMind Grow Landing Page}
    \label{fig:homepage}
\end{figure}

\begin{figure}[H]
    \centering
    \includegraphics[width=1.0\textwidth]{dashboard}
    \caption{User Dashboard with Feature Modules}
    \label{fig:dashboard}
\end{figure}

\subsection{Disease Prediction Interface}
The disease prediction page features a drag-and-drop zone for easy image upload.

\begin{figure}[H]
    \centering
    \includegraphics[width=0.8\textwidth]{prediction_interface}
    \caption{Disease Prediction Interface}
    \label{fig:prediction_ui}
\end{figure}

\subsection{Analysis Result}
Upon successful analysis, the system displays the disease name, confidence score, and detailed treatment recommendations.

\begin{figure}[H]
    \centering
    \includegraphics[width=1.0\textwidth]{analysis_result}
    \caption{Analysis Result Screen}
    \label{fig:result}
\end{figure}

\subsection{Invalid Image Handling}
The system correctly identifies and rejects non-plant images.

\begin{figure}[H]
    \centering
    \includegraphics[width=0.8\textwidth]{invalid_error}
    \caption{Invalid Image Error Display}
    \label{fig:invalid}
\end{figure}

\section{Model Performance Analysis}
The EfficientNet-B2 model was trained for 30 epochs. The training and validation loss curves indicate that the model converged well without significant overfitting.

\subsection{Training Metrics}
\begin{itemize}
    \item \textbf{Final Training Accuracy}: 98.45\%
    \item \textbf{Final Validation Accuracy}: 96.12\%
    \item \textbf{Training Loss}: 0.045
    \item \textbf{Validation Loss}: 0.123
\end{itemize}

\subsection{Class-wise Performance}
We evaluated the model on a held-out test set. The performance for key crops is summarized below.

\begin{table}[H]
    \centering
    \begin{tabular}{|l|c|c|c|}
    \hline
    \textbf{Class Name} & \textbf{Precision} & \textbf{Recall} & \textbf{F1-Score} \\ \hline
    Tomato - Bacterial Spot & 0.95 & 0.94 & 0.94 \\ \hline
    Tomato - Early Blight & 0.92 & 0.91 & 0.91 \\ \hline
    Tomato - Late Blight & 0.94 & 0.96 & 0.95 \\ \hline
    Potato - Early Blight & 0.96 & 0.95 & 0.95 \\ \hline
    Potato - Late Blight & 0.97 & 0.98 & 0.97 \\ \hline
    Corn - Common Rust & 0.99 & 0.99 & 0.99 \\ \hline
    Apple - Black Rot & 0.93 & 0.92 & 0.92 \\ \hline
    Grape - Black Rot & 0.95 & 0.96 & 0.95 \\ \hline
    \textbf{Overall Average} & \textbf{0.96} & \textbf{0.96} & \textbf{0.96} \\ \hline
    \end{tabular}
    \caption{Class-wise Performance Metrics}
    \label{tab:metrics}
\end{table}

\subsection{Confusion Matrix Analysis}
The confusion matrix reveals that the model sometimes confuses \textit{Tomato Early Blight} with \textit{Tomato Late Blight} due to the visual similarity of the lesions in their early stages. However, the distinction between healthy and diseased leaves is near perfect.

\begin{table}[H]
    \centering
    \renewcommand{\arraystretch}{1.5}
    \begin{tabular}{c|c|c|c|c|c|}
        \multicolumn{1}{c}{} & \multicolumn{5}{c}{\textbf{Predicted Class}} \\
        \cline{2-6}
        \textbf{Actual Class} & \textbf{T-Bact} & \textbf{T-Early} & \textbf{T-Late} & \textbf{P-Early} & \textbf{P-Late} \\
        \cline{2-6}
        \textbf{T-Bact} & \cellcolor{green!50} 95\% & \cellcolor{red!10} 2\% & \cellcolor{red!5} 1\% & 1\% & 1\% \\
        \cline{2-6}
        \textbf{T-Early} & \cellcolor{red!10} 3\% & \cellcolor{green!50} 91\% & \cellcolor{red!20} 5\% & 1\% & 0\% \\
        \cline{2-6}
        \textbf{T-Late} & \cellcolor{red!5} 1\% & \cellcolor{red!20} 4\% & \cellcolor{green!50} 94\% & 0\% & 1\% \\
        \cline{2-6}
        \textbf{P-Early} & 0\% & 1\% & 0\% & \cellcolor{green!50} 96\% & \cellcolor{red!15} 3\% \\
        \cline{2-6}
        \textbf{P-Late} & 0\% & 0\% & 1\% & \cellcolor{red!10} 2\% & \cellcolor{green!50} 97\% \\
        \cline{2-6}
    \end{tabular}
    \caption{Confusion Matrix Heatmap (Subset of Key Classes). T=Tomato, P=Potato.}
    \label{tab:confusion_matrix}
\end{table}

\section{Comparison with Existing Systems}
Compared to existing solutions like PlantVillage (MobileNet) or other academic implementations, AgroMind Grow offers:
\begin{enumerate}
    \item \textbf{Higher Accuracy}: EfficientNet-B2 outperforms MobileNetV2 by approximately 2-3\% on the same dataset.
    \item \textbf{Robustness}: The invalid image detection layer prevents false positives on non-plant images, a feature missing in most research prototypes.
    \item \textbf{Usability}: The integration of weather and market data makes it a more complete tool for farmers.
\end{enumerate}
