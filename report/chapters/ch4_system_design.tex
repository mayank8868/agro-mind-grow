\chapter{System Design}

\section{System Architecture}
The system follows a robust \textbf{Client-Server Architecture}, designed for scalability and maintainability. The architecture is decoupled, meaning the frontend and backend can be developed, deployed, and scaled independently.

\begin{enumerate}
    \item \textbf{Frontend (Client)}: Built with \textbf{React.js} and \textbf{Tailwind CSS}. It handles user interaction, image upload, and data visualization. It communicates with the backend via RESTful APIs.
    \item \textbf{Backend (Server)}: Built with \textbf{FastAPI (Python)}. It serves as the central controller. It handles:
    \begin{itemize}
        \item API Requests from the frontend.
        \item Image Pre-processing and Validation.
        \item Model Inference (loading the PyTorch model).
        \item Data Aggregation (fetching weather/market data).
    \end{itemize}
    \item \textbf{AI Engine}: A trained \textbf{EfficientNet-B2} model stored as a serialized file (\texttt{.pth}). It is loaded into memory by the backend to perform predictions.
    \item \textbf{Database}: A static JSON/Dictionary-based database (\texttt{disease\_database.py}) stores the static content for symptoms, causes, and treatments to ensure fast retrieval without database latency.
\end{enumerate}

\begin{figure}[H]
    \centering
    \includegraphics[width=1.0\textwidth]{system_architecture}
    \caption{Detailed System Architecture Diagram}
    \label{fig:sys_arch}
\end{figure}

\section{UML Diagrams}
Unified Modeling Language (UML) diagrams help in visualizing the system's design.

\subsection{Use Case Diagram}
The Use Case diagram depicts the interactions between the actors (Farmer, Admin) and the system.
\begin{itemize}
    \item \textbf{Actors}: Farmer (User), System Admin.
    \item \textbf{Use Cases}: Upload Image, View Prediction, Check Weather, View Market Prices, Login/Register.
\end{itemize}

\begin{figure}[H]
    \centering
    \includegraphics[width=0.9\textwidth]{use_case_diagram}
    \caption{Use Case Diagram}
    \label{fig:use_case}
\end{figure}

\subsection{Sequence Diagram}
The Sequence diagram illustrates the flow of messages for the "Disease Prediction" process.
\begin{enumerate}
    \item User uploads image $\rightarrow$ Frontend.
    \item Frontend sends POST request $\rightarrow$ Backend API.
    \item Backend validates image $\rightarrow$ Validation Module.
    \item Backend sends tensor $\rightarrow$ AI Model.
    \item AI Model returns probabilities $\rightarrow$ Backend.
    \item Backend fetches disease info $\rightarrow$ Database.
    \item Backend returns JSON response $\rightarrow$ Frontend.
    \item Frontend displays results $\rightarrow$ User.
\end{enumerate}

\begin{figure}[H]
    \centering
    \includegraphics[width=1.0\textwidth]{sequence_diagram}
    \caption{Sequence Diagram for Disease Prediction}
    \label{fig:sequence}
\end{figure}

\subsection{Activity Diagram}
The Activity diagram shows the workflow of the user interaction, including decision points like "Is Image Valid?" or "Is Confidence High?".

\begin{figure}[H]
    \centering
    \includegraphics[width=0.9\textwidth]{activity_diagram}
    \caption{Activity Diagram}
    \label{fig:activity}
\end{figure}

\section{Data Design}

\subsection{Disease Database Schema}
Although we use a dictionary-based storage for speed, the data follows a strict schema structure. This ensures consistency across all 38 disease classes.

\begin{lstlisting}[language=json, caption=JSON Structure of Disease Database]
{
  "Tomato___Bacterial_spot": {
    "symptoms": [
      "Small, water-soaked spots on leaves",
      "Spots turn brown and scabby",
      "Yellowing of leaves"
    ],
    "causes": [
      "Bacteria Xanthomonas campestris",
      "High humidity and warm temperatures",
      "Splashing rain or irrigation water"
    ],
    "treatments": {
      "chemical": [
        "Copper-based bactericides",
        "Streptomycin sulfate"
      ],
      "organic": [
        "Neem oil spray",
        "Copper soap"
      ],
      "prevention": [
        "Use disease-free seeds",
        "Crop rotation",
        "Avoid overhead irrigation"
      ]
    }
  },
  "Potato___Early_blight": {
    ...
  }
}
\end{lstlisting}

\section{Hardware / Software Specifications}

\subsection{Hardware Requirements}
\begin{table}[H]
    \centering
    \begin{tabular}{|l|l|l|}
    \hline
    \textbf{Component} & \textbf{Minimum Specification} & \textbf{Recommended Specification} \\ \hline
    Processor & Intel Core i3 / AMD Ryzen 3 & Intel Core i5 / AMD Ryzen 5 \\ \hline
    RAM & 4 GB & 8 GB or higher \\ \hline
    Storage & 10 GB HDD & 256 GB SSD \\ \hline
    GPU & Integrated Graphics & NVIDIA GTX/RTX Series (for Training) \\ \hline
    Internet & 2 Mbps & 10 Mbps or higher \\ \hline
    \end{tabular}
    \caption{Hardware Requirements}
    \label{tab:hardware}
\end{table}

\subsection{Software Requirements}
\begin{table}[H]
    \centering
    \begin{tabular}{|l|l|}
    \hline
    \textbf{Component} & \textbf{Specification} \\ \hline
    Operating System & Windows 10/11, Linux (Ubuntu), or macOS \\ \hline
    Frontend Technology & React.js, Vite, TypeScript, Tailwind CSS \\ \hline
    Backend Technology & Python 3.9+, FastAPI, Uvicorn \\ \hline
    AI/ML Libraries & PyTorch, Torchvision, NumPy, Pillow \\ \hline
    IDE / Tools & VS Code, Git, Postman \\ \hline
    \end{tabular}
    \caption{Software Requirements}
    \label{tab:software}
\end{table}
